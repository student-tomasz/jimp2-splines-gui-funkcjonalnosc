\documentclass[10pt,a4paper]{article}
\usepackage[a4paper]{geometry}

\usepackage{polski}
\usepackage{xltxtra}
\usepackage{relsize}
\usepackage{fancyvrb}
\usepackage[pdfborder={0 0 0}]{hyperref}
\usepackage{booktabs}

\defaultfontfeatures{Mapping=tex-text}
\setromanfont{Charis SIL}
%\setsansfont[Scale=MatchLowercase]{Gill Sans}
\setmonofont[Scale=MatchLowercase]{Menlo}
\linespread{1.25}

\DefineVerbatimEnvironment%
  {SmallVerbatim}%
  {Verbatim}{fontsize=\relsize{-0.5},numbers=left,numbersep=-10pt,frame=lines,tabsize=4}

\newcommand{\prog}[1]{\texttt{#1}}

\begin{document}

%%fakesection{Tytuł}
\title{ 
  Interpolacja funkcjami sklejanymi\\
  {\normalsize Specyfikacja funkcjonalna projektu nr 2}\\\vspace{-12pt}
  {\normalsize z przedmiotu \emph{Języki i metody programowania 2}}
}
\author{
  Tomasz Cudziło\\
  {\small EE PW, 211552}
}
\date{\today}
\maketitle

\section*{Zadanie}
\label{sec:zadanie}

Napisać program, z~grafikcznym interfejsem użytkownika, wyznaczający
współczynniki funkcji sklejanych trzeciego stopnia aproksymujących zadany ciąg
danych pomiarowych.

\vspace{24pt}

\section{Opis}
\label{sec:opis}

Poprzez GUI aplikacja pozwala na:
\begin{itemize}
  \item wczytanie z~pliku lub ręczne wprowadzenie danych wejściowych
    i~wyświetlenie wyliczonych funkcji na wykresie,
  \item wczytanie z~pliku wyliczonych funkcji sklejanych i~wyświetlenie ich na
    wykresie,
  \item edycję wczytanych punktów, zapis wprowadzonych zmian oraz eksport
    wyliczonych wielomianów,
  \item zoom i~scroll wykresu,
  \item eksport wykresu do pliku.
\end{itemize}

\section{Obsługa}
\label{sec:obsluga}

Aplikacja korzysta z~jednego okna głównego oraz kilku pomocniczych okien
dialogowych.

\subsection{Okno główne}

Okno główne zawiera z~lewej strony tabelę z~punktami. Jej zawartość można
edytować. W~prawej, górnej ćwiartce znajduje się pole na wykres. Pod wykresem
jest tabela z~wyznaczonymi wielomianami i~ich przedziałami.

Wykres można przybliżać korzystając z~ze skrótów klawiaturowych opisanych
w~tabeli \ref{tab:skroty} na stronie \pageref{tab:skroty}. Przesuwanie jest
dostępne wykorzystując paski na obrzeżach lub scroll myszki.

\begin{table}[p]
  \centering
  \begin{tabular}{r c}
    \toprule
    {\bf Akcja} & {\bf Skrót} \\
    \midrule
    zoom in     &  \prog{⌘+}  \\
    zoom out    &  \prog{⌘-}  \\
    zoom 100\%  &  \prog{⌘0}  \\
    \bottomrule
  \end{tabular}
  \caption{Skróty klawiaturowe aplikacji.}
  \label{tab:skroty}
\end{table}

\section{Pliki}
\label{sec:pliki}

\section{Wymagania}
\label{sec:wymagania}

Warunki potrzebne do uruchomienia aplikacji:
\begin{itemize}
  \item dostępna \prog{JVM},
  \item dostępny pakiet \prog{gnuplot} w wersji \prog{>= 4.4},
  \item binarka projektu nr~1 widoczna z~\prog{PATH}.
\end{itemize}

\end{document}
