\documentclass[10pt,a4paper]{article}
\usepackage[a4paper]{geometry}

\usepackage{polski}
\usepackage{xltxtra}
\usepackage{relsize}
\usepackage{alltt}
\usepackage{fancyvrb}
\usepackage[pdfborder={0 0 0}]{hyperref}

\defaultfontfeatures{Mapping=tex-text}
\setromanfont{Charis SIL}
%\setsansfont[Scale=MatchLowercase]{Gill Sans}
\setmonofont[Scale=MatchLowercase]{Menlo}
\linespread{1.25}

\DefineVerbatimEnvironment%
  {SmallVerbatim}%
  {Verbatim}{fontsize=\relsize{-0.5},numbers=left,numbersep=-10pt,frame=lines,tabsize=4}

\newcommand{\prog}[1]{\texttt{#1}}
\newcommand{\flag}[1]{\textbf{\prog{#1}}}

\begin{document}

%%fakesection{Tytuł}
\title{ 
  Interpolacja funkcjami sklejanymi\\
  {\normalsize Specyfikacja funkcjonalna projektu nr 2}\\\vspace{-12pt}
  {\normalsize z przedmiotu \emph{Języki i metody programowania 2}}
}
\author{
  Tomasz Cudziło\\
  {\small EE PW, 211552}
}
\date{\today}
\maketitle

\section*{Zadanie}
\label{sec:zadanie}

Napisać program, z~grafikcznym interfejsem użytkownika, wyznaczający
współczynniki funkcji sklejanych trzeciego stopnia aproksymujących zadany ciąg
danych pomiarowych.

\vspace{24pt}

\section{Opis}
\label{sec:opis}

\section{Obsługa}
\label{sec:obsluga}

\section{Pliki}
\label{sec:pliki}

\end{document}
